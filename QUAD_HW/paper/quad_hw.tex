\documentclass[a4paper,11pt]{article}
\usepackage[utf8]{inputenc}
\usepackage[italian]{babel}
\usepackage{graphicx}
\usepackage{subfigure}
\usepackage{amsmath}
\usepackage{bm}

\usepackage{graphicx}
\usepackage{xcolor}
\usepackage{colortbl}
\usepackage{fp}
\usepackage{tikz, pgfplots}
\usetikzlibrary{calc}
\usetikzlibrary{patterns}
\usetikzlibrary{intersections}
\usetikzlibrary{arrows}
\tikzset{>=latex}

% Color definition
\definecolor{green2}{RGB}{154, 205, 50}
\definecolor{green3}{RGB}{141, 182, 0}
\definecolor{blue2}{RGB}{0, 102, 255}
\definecolor{lightgreen}{RGB}{178,255,102}


%  Sistemazione margini
\addtolength{\hoffset}{-20pt}  % Riduco spazio bordo alto
\oddsidemargin = 10pt          % Riduco spazio dordo lati
\textwidth = 500pt             % Allargo parte scritta (righe) 
\textheight = 700pt            % Allargo parte scritta (colonne)

\title{\textbf{Extension to Quadrilateral Element of Three Field Hu-Washizu Elasticity Formulation Based on Biorthogonal Systems}}
\date{\today}
\author{Bishnu P. Lamichhane, M.Pingaro, P.Venini}

\begin{document}
\maketitle

\section{Linear elastic continuum problem}
In this section we briefly recovery the equations governing the linear elastic problem. 
The equilibrium equation is:
\begin{equation} \label{eq:equilibrio01}
- \mbox{div}(\bm{\sigma}) = \bm{f}\:,
\end{equation}
while in small deformation is:
\begin{equation}\label{eq:congruenza01}
\bm{d} = \bm{\varepsilon}(\bm{u}) = 
\frac{1}{2}(\bm{\nabla} \: \bm{u} + \bm{\nabla} \: \bm{u}^{T}) \:.
\end{equation}
In the case of linear elasticity we have:
\begin{equation} \label{eq:legame01}
\bm{\sigma} = \lambda \: tr(\bm{\varepsilon}) \bm{I} 
+ 2\mu \: \bm{\varepsilon}
\end{equation}
where $\mu$ and $\lambda$ are the Lamé constant.
By some algebra one obtains:
\begin{equation} \label{eq:legame02}
\bm{\sigma} =
\left(
\begin{array}{cc}
\lambda (\varepsilon_{11} + \varepsilon_{22}) & 0 \\
0 & \lambda (\varepsilon_{11} + \varepsilon_{22}) 
\end{array}
\right) +2\mu
\left(
\begin{array}{cc}
\varepsilon_{11} & \varepsilon_{12} \\
\varepsilon_{12} & \varepsilon_{22}
\end{array}
\right)\:,
\end{equation}
and rearranging the equation \eqref{eq:legame02}:
\begin{equation} \label{eq:legame03}
\bm{\sigma} =
\left(
\begin{array}{ll}
(\lambda+2\mu) \varepsilon_{11} + \lambda \: \varepsilon_{22} & 2\mu \: \varepsilon_{12} \\
2\mu \: \varepsilon_{12} & (\lambda+2\mu) \: \varepsilon_{22} + \lambda \: \varepsilon_{11}  \\
\end{array}
\right)\:.
\end{equation}

\section{Briefly introduction to modify Hu-Washizu}
We define the trial variables: $\bm{\varepsilon}(\bm{u})$, $\bm{d}$ and 
$\bm{\sigma}$, while the test variables are: $\bm{\varepsilon}(\bm{v})$, 
$\bm{e}$ and $\bm{\tau}$. 
\begin{equation}
-\int_{\Omega} \mbox{div} ( \bm{C} : \bm{d}) )\cdot \bm{v} = \bm{f}
\end{equation}

\begin{eqnarray}
& a( (\bm{u},\bm{d}), (\bm{v}, \bm{e}) ) 
+ b( (\bm{v}, \bm{e}), \bm{\sigma} ) &= l(\bm{v}) \\
& b( (\bm{u},\bm{d}), \bm{\tau} )                                 &= 0
\end{eqnarray}
where:
\begin{eqnarray}
& a( (\bm{u},\bm{d}), (\bm{v}, \bm{e}) ) 
&= \int_{\Omega} \bm{d} : (\bm{C}:\bm{e}) \: dx 
+ \alpha \int_{\Omega} (\bm{\varepsilon}(\bm{u}) 
- \bm{d}):(\bm{\varepsilon}(\bm{v}) - \bm{e}) \: dx \:, \\
& b( (\bm{u},\bm{d}), \bm{\tau} ) &= 
\int_{\Omega} (\bm{\varepsilon}(\bm{u}) - \bm{d}) : \bm{\tau} \: dx \:.
\end{eqnarray}
The modify weak formulation of the problem is:
\begin{equation} \label{eq:system_1}
\left\lbrace
\begin{array}{lll}
\alpha\int_{\Omega}(\bm{\varepsilon}(\bm{u})-\bm{d}):
\bm{\varepsilon}(\bm{v})\:dx 
+\int_{\Omega}\bm{\varepsilon}(\bm{v}):\bm{\sigma}\:dx &=\int_{\Omega}
\bm{f}\cdot\bm{v}\:dx \\
\int_{\Omega}\bm{d}:\bm{C}\bm{e}\:dx-\alpha\int_{\Omega}
(\bm{\varepsilon}(\bm{u})-\bm{d}):\bm{e}\:dx 
-\int_{\Omega}\bm{e}:\bm{\sigma}\:dx &=0 \\
\int_{\Omega}(\bm{\varepsilon}(\bm{u})-\bm{d}):\bm{\tau}\:dx &= 0 \\
\end{array}
\right.
\end{equation}
by rearranging:
\begin{equation} \label{eq:system_2}
\left\lbrace
\begin{array}{lll}
\alpha\int_{\Omega}\bm{\varepsilon}(\bm{u}):\bm{\varepsilon}(\bm{v})\:dx 
-\alpha\int_{\Omega}\bm{d}:\bm{\varepsilon}(\bm{v})\:dx
+\int_{\Omega}\bm{\varepsilon}(\bm{v}):\bm{\sigma}\:dx 
&=\int_{\Omega}\bm{f}\cdot\bm{v}\:dx\\ 
-\alpha\int_{\Omega}\bm{\varepsilon}(\bm{u}):\bm{e}\:dx
+\int_{\Omega}\bm{d}:\bm{C}\bm{e}\:dx
+\alpha\int_{\Omega}\bm{d}:\bm{e}\:dx
-\int_{\Omega}\bm{e}:\bm{\sigma}\:dx&= 0 \\
\int_{\Omega}\bm{\varepsilon}(\bm{u}):\bm{\tau}\:dx
-\int_{\Omega}\bm{d}:\bm{\tau}\:dx &= 0 \\
\end{array}
\right.
\end{equation}
It is possible to rewrite the system in equation \eqref{eq:system_2} in matrix form in the following way: 
\begin{equation}
\left[
\begin{array}{ccc}
\alpha \bm{A} & -\alpha \bm{B} & \bm{W} \\
-\alpha \bm{B}^{T} & \bm{K}+\alpha \bm{M} & -\bm{D} \\
\bm{W}^{T}  & - \bm{D}^{T} & \bm{0} \\ 
\end{array}
\right] 
\left[
\begin{array}{c}
\bm{x}_{u} \\
\bm{x}_{d} \\
\bm{x}_{\sigma} \\
\end{array}
\right] =
\left[
\begin{array}{c}
\bm{b}_{f} \\
\bm{0} \\
\bm{0} \\
\end{array}
\right]\:,
\end{equation}
where 
$\bm{A}=\int_{\Omega}\bm{\varepsilon}(\bm{u}):\bm{\varepsilon}(\bm{v})$, 
$\bm{B}=\int_{\Omega}\bm{d}:\bm{\varepsilon}(\bm{v})$, 
$\bm{W}=\int_{\Omega}\bm{\sigma}:\bm{\varepsilon}(\bm{v})$, 
$\bm{K}=\int_{\Omega}\bm{C}\bm{e}:\bm{d}$, 
$\bm{M}=\int_{\Omega}\bm{e}:\bm{d}$, 
$\bm{D}=\int_{\Omega}\bm{\sigma}:\bm{e}$.  
$\bm{D}$ is a diagonal matrix.
Using this property it is possible condense statically $\bm{x}_{d}$ and $\bm{x}_{\sigma}$, and we obtain the following system in the only unknown 
$\bm{x}_{u}$:
\begin{equation}
\left[ \alpha \bm{A} - \alpha \left( \bm{B} \bm{D}^{-1} \bm{W}^{-T} 
+ \bm{W} \bm{D}^{-1} \bm{B}^{T} \right) + 
\bm{W} \bm{D}^{-1} \left( \bm{K} + \alpha \bm{M} \right) 
\bm{D}^{-1} \bm{W}^{T} \right] \bm{x}_{u} = \bm{b}_{f}
\end{equation}

\section{Finite element discretization}
We consider a quasi-uniform triangulation $\mathcal{T}_{h}$ of the polygonal domain $\Omega$ consists of simply, either quadrilateral or hexahedral.
We take into account of standard bilinear finite element space $K_{h}\subset H^{1}(\Omega)$ defined on the triangulation $\mathcal{T}_{h}$, where:
\begin{equation}
K_{h} := \left\lbrace v \in C^{0}(\Omega): v_{\mid T}\in \mathcal{Q}_{1}(T), \: T\in \mathcal{T}_{h} \right\rbrace ,
\:\:\:\:\: K_{h}^{0} = K_{h} \cap H_{0}^{1}(\Omega) ,
\end{equation} 
and the space of bubble functions
\begin{equation}
B_{h} := \left\lbrace b_{T}\in H^{1}(T): b_{T\mid \partial T}=0 
\mbox{ and } \int_{T} b_{T} \: dx > 0, \: T\in \mathcal{T}_{h} 
\right\rbrace ,
\end{equation}
and we define the spaces for strain and displacement as $\bm{S}_{h}:=[K_{h}]^{2\times 2}$ and $\bm{V}_{h}:=\left[K_{h}^{0} \bigoplus B_{h}\right]^{2}$. 
In the next section we discuss the different choosing of bubble functions.
For the discrete stress space we use:
\begin{equation}
\bm{M}_{h}:=\left\lbrace \bm{\tau}_{h}\in \left[M_{h}\right]^{2\times 2}: \: \int_{\Omega} \bm{\tau}_{h} : \bm{1} \: dx = 0 \right\rbrace \subset \bm{S}_{0}\:,
\end{equation}
and let $\left\lbrace \phi_{1}, \cdots, \phi_{n}\right\rbrace$ and 
$\left\lbrace \mu_{1}, \cdots, \mu_{n}\right\rbrace$ the $n$ the basis functions for the space $V_{h}$ and $M_{h}$ respectively, we construct the functions $\mu_{i}$ using the following biorthogonality property between the space $V_{h}$ and $M_{h}$:
\begin{equation}
\int_{\Omega}\mu_{i}\phi_{j} \: dx = c_{j}\delta_{ij}\:, \:\: c_{j}\neq 0\:,
\:\: 1 \leq i,j \leq n \:,
\end{equation}
where $\delta_{ij}$ is Kronecker symbol, and $c_{j}$ is a scaling factor which can be chosen to be proportion al to the area of support of $\phi_{j}$.
The local basis function of $K_{h}$ and $M_{h}$ for the reference square element (see figure \ref{fig:ref_element}) $\hat{T}:=\left\lbrace (x,y): -1\leq x \leq 1 \:, -1\leq y \leq 1 \right\rbrace$ are:
\begin{figure}[h!]
\begin{center}
\include{figure/reference_element}
\caption{Reference Element \label{fig:ref_element}}
\end{center}
\end{figure}
%
\begin{equation}
\begin{split}
&\phi_{1}=\frac{1}{4}(1-x)(1-y) \:, \:\:\:\: \phi_{2}=\frac{1}{4}(1+x)(1-y) \:,\\
&\phi_{3}=\frac{1}{4}(1+x)(1+y) \:, \:\:\:\: \phi_{4}=\frac{1}{4}(1-x)(1+y) \:.
\end{split}
\end{equation}   
and
\begin{equation}
\begin{split}
&\mu_{1} = 1 - 3x - 3y + 9xy \:, \:\:\:\: \mu_{2} = 1 + 3x - 3y - 9xy \:,\\
&\mu_{3} = 1 + 3x + 3y + 9xy \:, \:\:\:\: \mu_{4} = 1 - 3x + 3y + 9xy \:.
\end{split}
\end{equation}
It is important to observe that the global basis functions of the space $M_{h}$ are not continuous.
   
\section{Bubble functions}
In this section we detail the different choosing of the bubble functions. Addition of the bubble functions is essential to create a stable space. we have four types of bubbles.
In the first two cases we use a modification of the standard bubble function ,that is for the reference element:
\begin{equation}\label{eq:standard_bubble_fun}
b_{T}(x,y) = (1-x^{2})(1-y^{2})\:,
\end{equation}
while in the next two, we add to the standard bubble function another one.
\subsection{One Bubble function (type 1)}
As a first choice of bubble function we use:
\begin{equation}\label{eq:bubble_1}
\hat{b}_{T}(x,y) = c_{T}\cdot\phi_{T}(x,y)\cdot b_{T}(x,y)\:, 
\end{equation}
where $c_{T}$ is a coefficient in order to obtain $\hat{b}_{T}(x_{g},y_{g}) = 1$ (where $\bm{g}$ is the centroid of the elements), $\phi_{K}$ is the standard bilinear basis function corresponding to the lower-left corner of the square $T$.
In the case of reference square element we obtain:
\begin{equation} \label{eq:bubble_1_ref}
\hat{b}_{T}(x,y) = (1-x)(1-y)(1-x^{2})(1-y^{2})\:.
\end{equation}

\subsection{One Bubble function (type 2)}
The second choice of bubble function we take:
\begin{equation}\label{eq:bubble_2}
\hat{b}_{T}(x,y) = c_{T}\cdot (a+bx+cy)\cdot b_{T}(x,y)\:,
\end{equation}
where $a,b,c \in \mathrm{R}$ and $a,b,c \neq 0$.
For simplicity we set $a=b=c=1$ and we obtain for the reference square:
\begin{equation}\label{eq:bubble_2_ref}
\hat{b}_{T}(x,y) = (1+x+y)(1-x^{2})(1-y^{2})\:.
\end{equation} 

\subsection{Two Bubble functions}
Using two bubble functions, where the first is the standard bubble function and the second bubble is a modification of the standard bubble:
\begin{equation}\label{eq:bubble_3}
\begin{split}
&\hat{b}_{T1}(x,y) = b_{T}\:, \\
&\hat{b}_{T2}(x,y) = c_{T}\cdot (ax+by)\cdot b_{T}\:,
\end{split}
\end{equation}
where $a,b \in \mathrm{R}$ and $a^{2}+b^{2}\neq 0$.
For the sake of simplicity we adopt $a=b=1$.
One obtains:
\begin{equation}\label{eq:bubble_3_ref}
\begin{split}
&\hat{b}_{T1}(x,y) = (1-x^{2})(1-y^{2})\:, \\
&\hat{b}_{T2}(x,y) = (x+y)(1-x^{2})(1-y^{2})\:.
\end{split}
\end{equation}

\subsection{Two Bubble functions, which one mixed}
As a finally choice of bubbles we use a standard bubble function plus one mixed bubble function for the two components of displacement.
\begin{equation}
\begin{split}
&\hat{b}_{T1}(x,y) = b_{T}\:, \\
&\hat{b}_{T2,x}(x,y) = (\nabla \phi_{1})_{x}\cdot b_{T}\:, \\
&\hat{b}_{T2,y}(x,y) = (\nabla \phi_{1})_{y}\cdot b_{T}\:,
\end{split}
\end{equation}
where $(\nabla\phi_{1})_{i}$ is $i$-th component of the gradient of the first shape function $\phi$.
In this way we have as shape function for the displacement using the mixed bubble function the vector $\left[\hat{b}_{T2,x}(x,y),\hat{b}_{T2,y}(x,y)\right]$. 

\section{Numerical example}
In this section we report some examples using the presented formulation to proven the good behaviour. 

\subsection{Square problem}
First example is a unit square domain with homogeneous Dirichlet boundary conditions.
The Lamé constant are fix to $\lambda = 123$ and $\mu=79.3$.
By imposition of the previously exact solution one obtain for the body force $f$
\begin{equation}
\begin{split}
&f_{1} = -\pi^{2} \cos(\pi x) \sin(\pi y) \left( \lambda + \mu + 2\lambda\cos(\pi y) + 
12\mu\cos(\pi y)\right), \\
&f_{2} = -\pi^{2}\sin(\pi x)\left( \lambda\cos(\pi y) + 3\mu\cos(\pi y) + 2\lambda\left(2\cos(\pi y)^{2} 
- 1\right) + 2\mu\left(2\cos(\pi y)^{2} - 1\right) \right)
\end{split}
\end{equation}
The exact solution is
\begin{equation} \label{eq:exact_solution}
u_{1} = \cos (\pi x) \sin(2\pi y), \mbox{ } u_{2} = \sin(\pi x)\cos(\pi y). 
\end{equation} 
The problem is study using two type of mesh: first of all using a square mesh and before using a trapezoidal mesh. The two types of mesh are shown in figures \ref{fig:square_regular} and \ref{fig:square_irregular}.
Figures \ref{fig:error_l2_1B_type1}, \ref{fig:error_l2_1B_type2}, \ref{fig:error_l2_2B} and \ref{fig:error_l2_2B_mixed} shown the error in norm $L^{2}$ in the case of regular mesh for the different types of bubble functions used and types of coefficient $\alpha$. All types of element converge in a good way.
In Figures  \ref{fig:error_dist_l2_1B_type1}, \ref{fig:error_dist_l2_1B_type2}, \ref{fig:error_dist_l2_2B} and \ref{fig:error_dist_l2_2B_mixed} we report the previously results in the case of trapezoidal meshes. 
%
\begin{figure}[h!]
\begin{center}
\subfigure[Regular mesh \label{fig:square_regular}]
{\input{figure/square_regular}}
\hspace{5pt}
\subfigure[Trapezoidal mesh \label{fig:square_irregular}]{\input{figure/square_trapezoidal}}
\caption{Square Problem}
\end{center}
\end{figure}
% Error L2
\begin{figure}[h!]
\begin{center}
\subfigure[type 1 \label{fig:error_l2_1B_type1}]
{\input{square_example/bolla1_type1/elastic_error_1B_l2}}
\subfigure[type 2 \label{fig:error_l2_1B_type2}]
{\input{square_example/bolla1_type2/elastic_error_1B_l2}}
\caption{The relative error vs. the number of elements measured relative 
to the $L^{2}$ norm (Case one bubble function and regular mesh)}
\end{center}
\end{figure}
%
\begin{figure}[h!]
\begin{center}
\subfigure[Case two bubble function \label{fig:error_l2_2B}]{\input{square_example/bolla2/elastic_error_2B_l2}}
\subfigure[Case two bubble function of which one mixed \label{fig:error_l2_2B_mixed}]{%\documentclass{article}
%
%\usepackage{graphicx}
%\usepackage{xcolor}
%\usepackage{colortbl}
%\usepackage{fp}
%\usepackage{tikz, pgfplots}
%\usetikzlibrary{calc}
%\usetikzlibrary{patterns}
%\usetikzlibrary{intersections}
%\usetikzlibrary{arrows}
%\tikzset{>=latex}
%
%% Color definition
%\definecolor{green2}{RGB}{154, 205, 50}
%\definecolor{green3}{RGB}{141, 182, 0}
%\definecolor{blue2}{RGB}{0, 102, 255}
%\definecolor{lightgreen}{RGB}{178,255,102}
%
%
%\begin{document}
%
%\begin{figure}[!h]
%\begin{center}
\begin{tikzpicture}
 %transpose legend
 \begin{loglogaxis}[width=0.45\textwidth, height=0.45\textwidth,
  legend style={anchor=south, at={(0.5, 1.07)}, draw=none, font=\scriptsize},
  grid=major,
  legend columns=3,
  xlabel={\large Number of Elements},
  ylabel={\Large $\frac{\parallel u_{h}-u \parallel_{L^{2}}}{\parallel u \parallel_{L^{2}}}$},
  xmin=10, xmax=1e4,
  ymin=1e-3, ymax=1e0,
  ytick={1e-3, 1e-2, 1e-1, 1},
  yticklabels={$10^{-3}$, $10^{-2}$, $10^{-1}$, $10^{0}$},
  xtick={1e1, 1e2, 1e3, 1e4},
  xticklabels={$10^{1}$, $10^{2}$, $10^{3}$, $10^{4}$},
 ]
 % Reference
 %\addplot[black, ultra thick] coordinates{
 %(50, 1.8248e-5)
 %(15000, 1.8248e-5)
 %}; 
 %\addlegendentry{Reference} 
 %
 \addplot[red, mark=+, thick, mark options={solid}]
 table [x index={0}, y index={1}]
 {square_example/bolla2_mixed/elastic_error_disp_u_l2_lmu1.txt};
 \addlegendentry{$\alpha / \mu=1$}
 %
 \addplot[blue, mark=x, thick, mark options={solid}]
 table [x index={0}, y index={1}]
 {square_example/bolla2_mixed/elastic_error_disp_u_l2_lmu2.txt};
 \addlegendentry{$\alpha / \mu=2$}
 %
 \addplot[green3, mark=o, thick, mark options={solid}]
 table [x index={0}, y index={1}]
 {square_example/bolla2_mixed/elastic_error_disp_u_l2_lmu3.txt};
 \addlegendentry{$\alpha / \mu=3$}
 
% \addplot[red, mark=+, very thick]
% table [x index={0}, y index={1}]
% {square_example/bolla2_mixed/elastic_error_dist_disp_u_l2_lmu1.txt};
% \addlegendentry{Dist $\alpha / \mu=1$}
 %
 %\addplot[blue, mark=x, very thick]
 %table [x index={0}, y index={1}]
 %{};
 %\addlegendentry{}
 %
 %\addplot[green3, mark=x, very thick]
 %table [x index={0}, y index={1}]
 %{};
 %\addlegendentry{}
 % 
 \end{loglogaxis}
\end{tikzpicture}
%\end{center}
%\caption{The relative error versus the number of elements measured relative 
%to the $L^{2}$}
%\end{figure}
%
%\end{document}
}
\caption{The relative error versus the number of elements measured relative to the $L^{2}$ norm (regular mesh)}
\end{center}
\end{figure}
% Error L2 (Distorted)
\begin{figure}[h!]
\begin{center}
\subfigure[type 1 \label{fig:error_dist_l2_1B_type1}]
{\input{square_example/bolla1_type1/elastic_error_dist_1B_l2.tex}}
\subfigure[type 2 \label{fig:error_dist_l2_1B_type2}]
{\input{square_example/bolla1_type2/elastic_error_dist_1B_l2.tex}}
\caption{The relative error vs. the number of elements measured relative 
to the $L^{2}$ norm (Case one bubble function and Trapezoidal mesh)}
\end{center}
\end{figure}
%
\begin{figure}[h!]
\begin{center}
\subfigure[Case two bubble function \label{fig:error_dist_l2_2B}]
{\input{square_example/bolla2/elastic_error_dist_2B_l2.tex}}
\subfigure[Case two bubble function of which one mixed \label{fig:error_dist_l2_2B_mixed}]
{\input{square_example/bolla2_mixed/elastic_error_dist_2B_mixed_l2.tex}}
\caption{The relative error vs. the number of elements measured relative 
to the $L^{2}$ norm (Case one bubble function and Trapezoidal mesh)}
\end{center}
\end{figure}
%

\subsection{Cantilever beam problem}
Now we consider the beam with length $L=10$ and height $l=2$ as we shown in figure (). The Young modulus is set equal to $E=1500$ and the Poisson $\nu=0.4999$ and subjected to a distributed load as in figure \ref{fig:beam_geometry} with $f=300$.
%
\begin{figure}[h!]
\begin{center}
\input{figure/beam}
\caption{Beam cantilever geometry \label{fig:beam_geometry}}
\end{center}
\end{figure}
%
The exact solution is:
\begin{equation}
\begin{split}
&u(x,y) = \frac{2 f}{E l} (1-\nu^{2}) x \left( \frac{l}{2} - y \right)\: , \\
&v(x,y) = \frac{f}{E l} \left[ x^{2} + \frac{\nu}{1-\nu}\left(y^{2}-l y 
\right) \right] \: .
\end{split}
\end{equation}
We use to model the beam two types of mesh: regular anf trapezoidal as in the previously example (see figures \ref{fig:square_regular} and \ref{fig:square_irregular}).
%
\begin{figure}[h!]
\begin{center}
\subfigure[Type 1 \label{fig:beam_reg_1B_type1}]
{\input{beam_example/regular/bolla1/elastic_error_1B_type_1_l2}}
\subfigure[Type 2 \label{fig:beam_reg_1B_type2}]
{\input{beam_example/regular/bolla1/elastic_error_1B_type_2_l2}}
\caption{Beam Cantilever: the relative error vs. the number of elements measured relative to the $L^{2}$ norm (regular mesh)}
\end{center}
\end{figure}
%
\begin{figure}[h!]
\begin{center}
\subfigure[Case two bubble function \label{fig:beam_reg_2B}]
{\input{beam_example/regular/bolla2/elastic_error_2B_l2}}
\subfigure[Case two bubble function of which one mixed \label{fig:beam_reg_2B_mixed}]
{%\documentclass{article}
%
%\usepackage{graphicx}
%\usepackage{xcolor}
%\usepackage{colortbl}
%\usepackage{fp}
%\usepackage{tikz, pgfplots}
%\usetikzlibrary{calc}
%\usetikzlibrary{patterns}
%\usetikzlibrary{intersections}
%\usetikzlibrary{arrows}
%\tikzset{>=latex}
%
%% Color definition
%\definecolor{green2}{RGB}{154, 205, 50}
%\definecolor{green3}{RGB}{141, 182, 0}
%\definecolor{blue2}{RGB}{0, 102, 255}
%\definecolor{lightgreen}{RGB}{178,255,102}
%
%
%\begin{document}
%
%\begin{figure}[!h]
%\begin{center}
\begin{tikzpicture}
 %transpose legend
 \begin{loglogaxis}[width=0.45\textwidth, height=0.45\textwidth,
  legend style={anchor=south, at={(0.5, 1.07)}, draw=none, font=\scriptsize},
  grid=major,
  legend columns=3,
  xlabel={\large Number of Elements},
  ylabel={\Large $\frac{\parallel u_{h}-u \parallel_{L^{2}}}{\parallel u \parallel_{L^{2}}}$},
  xmin=10, xmax=1e4,
  ymin=1e-3, ymax=1e0,
  ytick={1e-3, 1e-2, 1e-1, 1},
  yticklabels={$10^{-3}$, $10^{-2}$, $10^{-1}$, $10^{0}$},
  xtick={1e1, 1e2, 1e3, 1e4},
  xticklabels={$10^{1}$, $10^{2}$, $10^{3}$, $10^{4}$},
 ]
 % Reference
 %\addplot[black, ultra thick] coordinates{
 %(50, 1.8248e-5)
 %(15000, 1.8248e-5)
 %}; 
 %\addlegendentry{Reference} 
 %
 \addplot[red, mark=+, thick, mark options={solid}]
 table [x index={0}, y index={1}]
 {square_example/bolla2_mixed/elastic_error_disp_u_l2_lmu1.txt};
 \addlegendentry{$\alpha / \mu=1$}
 %
 \addplot[blue, mark=x, thick, mark options={solid}]
 table [x index={0}, y index={1}]
 {square_example/bolla2_mixed/elastic_error_disp_u_l2_lmu2.txt};
 \addlegendentry{$\alpha / \mu=2$}
 %
 \addplot[green3, mark=o, thick, mark options={solid}]
 table [x index={0}, y index={1}]
 {square_example/bolla2_mixed/elastic_error_disp_u_l2_lmu3.txt};
 \addlegendentry{$\alpha / \mu=3$}
 
% \addplot[red, mark=+, very thick]
% table [x index={0}, y index={1}]
% {square_example/bolla2_mixed/elastic_error_dist_disp_u_l2_lmu1.txt};
% \addlegendentry{Dist $\alpha / \mu=1$}
 %
 %\addplot[blue, mark=x, very thick]
 %table [x index={0}, y index={1}]
 %{};
 %\addlegendentry{}
 %
 %\addplot[green3, mark=x, very thick]
 %table [x index={0}, y index={1}]
 %{};
 %\addlegendentry{}
 % 
 \end{loglogaxis}
\end{tikzpicture}
%\end{center}
%\caption{The relative error versus the number of elements measured relative 
%to the $L^{2}$}
%\end{figure}
%
%\end{document}
}
\caption{Beam Cantilever: the relative error vs. the number of elements measured relative to the $L^{2}$ norm (regular mesh)} 
\end{center}
\end{figure}
we shown in figures \ref{fig:beam_trap_1B_type1}, \ref{fig:beam_trap_1B_type2}, \ref{fig:beam_trap_2B} and \ref{fig:beam_trap_2B_mixed} the $L^{2}$-norm error for different types of bubble functions used in the case of $\alpha/mu:={1,2,3}$, while in figures \ref{fig:beam_trap_1B_type1}, \ref{fig:beam_trap_1B_type2}, \ref{fig:beam_trap_2B} and \ref{fig:beam_trap_2B_mixed} the same plots using trapezoidal meshes.
%% Trapezoidal mesh
\begin{figure}[h!]
\begin{center}
\subfigure[Type 1 \label{fig:beam_trap_1B_type1}]
{\input{beam_example/trapezoidal/bolla1/elastic_error_1B_dist_type_1_l2}}
\subfigure[Type 2 \label{fig:beam_trap_1B_type2}]
{%\documentclass{article}
%
%\usepackage{graphicx}
%\usepackage{xcolor}
%\usepackage{colortbl}
%\usepackage{fp}
%\usepackage{tikz, pgfplots}
%\usetikzlibrary{calc}
%\usetikzlibrary{patterns}
%\usetikzlibrary{intersections}
%\usetikzlibrary{arrows}
%\tikzset{>=latex}
%
%% Color definition
%\definecolor{green2}{RGB}{154, 205, 50}
%\definecolor{green3}{RGB}{141, 182, 0}
%\definecolor{blue2}{RGB}{0, 102, 255}
%\definecolor{lightgreen}{RGB}{178,255,102}
%
%
%\begin{document}
%\begin{figure}[!h]
%\begin{center}
\begin{tikzpicture}
 \begin{loglogaxis}[width=0.45\textwidth, height=0.45\textwidth,
  legend style={anchor=south, at={(0.5, 1.07)}, draw=none, font=\scriptsize},
  grid=major,
  legend columns=3, transpose legend,
  xlabel={Number of Elements},
  ylabel={$\left( \parallel u_{h}-u \parallel_{L^{2}}\right)/ \parallel u \parallel_{L^{2}}$},
  xmin=1, xmax=1e4,
  ymin=1e-3, ymax=1e0,
  ytick={1e-3, 1e-2, 1e-1, 1},
  yticklabels={$10^{-3}$, $10^{-2}$, $10^{-1}$, $10^{0}$},
  xtick={1e0, 1e1, 1e2, 1e3, 1e4},
  xticklabels={$10^{0}$, $10^{1}$, $10^{2}$, $10^{3}$, $10^{4}$},
 ]
 % Reference
 %\addplot[black, ultra thick] coordinates{
 %(50, 1.8248e-5)
 %(15000, 1.8248e-5)
 %}; 
 %\addlegendentry{Reference} 
 %
 \addplot[red, mark=+, thick, mark options={solid}]
 table [x index={0}, y index={1}]
 {beam_example/trapezoidal/bolla1/error_beam_u_l2_type_2_dist_1mu.txt};
 \addlegendentry{$\alpha / \mu=1$}
 %
 \addplot[blue, mark=x, thick, mark options={solid}]
 table [x index={0}, y index={1}]
 {beam_example/trapezoidal/bolla1/error_beam_u_l2_type_2_dist_2mu.txt};
 \addlegendentry{$\alpha / \mu=2$}
 %
 \addplot[green3, mark=o, thick, mark options={solid}]
 table [x index={0}, y index={1}]
 {beam_example/trapezoidal/bolla1/error_beam_u_l2_type_2_dist_3mu.txt};
 \addlegendentry{$\alpha / \mu=3$}
 %
 %\addplot[red, mark=+, very thick]
 %table [x index={0}, y index={1}]
 %{};
 %\addlegendentry{}
 %
 %\addplot[blue, mark=x, very thick]
 %table [x index={0}, y index={1}]
 %{};
 %\addlegendentry{}
 %
 %\addplot[green3, mark=x, very thick]
 %table [x index={0}, y index={1}]
 %{};
 %\addlegendentry{}
 % 
 \end{loglogaxis}
\end{tikzpicture}
%\end{center}
%\caption{The relative $L^{2}$ error vs. the number of elements (mesh trapezoidal)}
%\end{figure}
%
%\end{document}
}
\caption{Beam Cantilever: the relative error vs. the number of elements measured relative to the $L^{2}$ norm (trapezoidal mesh)}
\end{center}
\end{figure}
%
\begin{figure}[h!]
\begin{center}
\subfigure[Case two bubble function \label{fig:beam_trap_2B}]
{\input{beam_example/trapezoidal/bolla2/elastic_error_2B_dist_l2}}
\subfigure[Case two bubble function of which one mixed 
\label{fig:beam_trap_2B_mixed}]
{\input{beam_example/trapezoidal/bolla2_mixed/elastic_error_2B_mixed_dist_l2}}
\caption{Beam Cantilever: the relative error vs. the number of elements measured relative to the $L^{2}$ norm (trapezoidal mesh)}
\end{center}
\end{figure}
In the all cases the elements distorted have a good behaviour respect to the regular mesh.  

\subsection{Cook's membrane}
The final example is the Cook's membrane. That is a typical benchmark and consist of a beam with vertex: $(0,0)$, $(48,44)$, $(48,60)$ and $(0,44)$.
The left vertical edge is clamped and the right vertical edge subjected to the vertical distributed forces with resultant $F=100$ as it shown in figure \ref{fig:cook}.
%
\begin{figure}[h!]
\begin{center}
\input{figure/cook_membrane}
\caption{Cook's Membrane geometry \label{fig:cook}}
\end{center}
\end{figure}
The material properties are taken to be $E = 250$ and $\nu = 0.4999$, so that a nearly incompressible response is obtained.
We report in figures \ref{fig:cook_alpha_1}, \ref{fig:cook_alpha_1mu}, \ref{fig:cook_alpha_2mu} and \ref{fig:cook_alpha_3mu} the vertical displacement of the point $A$ versus the number of element per side for different choosing of the parameter $\alpha=\left\lbrace 1, \mu, 2\mu, 3\mu \right\rbrace$.
% Alpha 1, 1*mu, 2*mu, 3*mu
\begin{figure}[h!]
\begin{center}
\subfigure[$\alpha=1$ \label{fig:cook_alpha_1}]
{%\documentclass{article}
%
%\usepackage{graphicx}
%\usepackage{xcolor}
%\usepackage{colortbl}
%\usepackage{fp}
%\usepackage{tikz, pgfplots}
%\usetikzlibrary{calc}
%\usetikzlibrary{patterns}
%\usetikzlibrary{intersections}
%\usetikzlibrary{arrows}
%\tikzset{>=latex}
%
%% Color definition
%\definecolor{green2}{RGB}{154, 205, 50}
%\definecolor{green3}{RGB}{141, 182, 0}
%\definecolor{blue2}{RGB}{0, 102, 255}
%\definecolor{lightgreen}{RGB}{178,255,102}
%
%
%\begin{document}
%
%\begin{figure}[!h]
%\begin{center}
\begin{tikzpicture}
 %transpose legend
 \begin{axis}[width=0.45\textwidth, height=0.45\textwidth,
  legend style={anchor=south, at={(0.5, 1.07)}, draw=none, font=\scriptsize},
  legend columns=2,
  xlabel={Number of Elements per Side},
  ylabel={Vertical displacement of point A},
  xmin=0, xmax=32,
  ymin=0, ymax=8,
  ytick={0, 1, 2, 3, 4, 5, 6, 7, 8},
  yticklabels={$0$, $1$, $2$, $3$, $4$, $5$, $6$, $7$, $8$},
  xtick={0, 2, 4, 8, 16, 32},
  xticklabels={$0$, $2$, $4$, $8$, $16$, $32$},
 ]
 % Reference
 %\addplot[black, ultra thick] coordinates{
 %(50, 1.8248e-5)
 %(15000, 1.8248e-5)
 %}; 
 %\addlegendentry{Reference} 
 %
 \addplot[red, mark=+, thick, dashed, mark options={solid}]
 table [x index={0}, y index={1}]
 {quad_cook_membrane_graph/alpha_2mu/cook_1B_type_1.txt};
 \addlegendentry{1 Bubble type 1}
 %
 \addplot[red, mark=+, thick, mark options={solid}]
 table [x index={0}, y index={1}]
 {quad_cook_membrane_graph/alpha_2mu/cook_1B_type_2.txt};
 \addlegendentry{1 Bubble type 2}
 %
 \addplot[green3, mark=o, thick, mark options={solid}]
 table [x index={0}, y index={1}]
 {quad_cook_membrane_graph/alpha_2mu/cook_2B.txt};
 \addlegendentry{2 Bubble}
 %
 \addplot[blue, mark=x, thick, mark options={solid}]
 table [x index={0}, y index={1}]
 {quad_cook_membrane_graph/alpha_2mu/cook_2B_mixed.txt};
 \addlegendentry{2 Bubble Mixed}
 %
 %\addplot[blue, mark=x, very thick]
 %table [x index={0}, y index={1}]
 %{};
 %\addlegendentry{}
 %
 %\addplot[green3, mark=x, very thick]
 %table [x index={0}, y index={1}]
 %{};
 %\addlegendentry{}
 % 
 \end{axis}
\end{tikzpicture}
%\end{center}
%\caption{Vertical Displacement of point A vs. the number of element per side ($\alpha / \mu=2$)}
%\end{figure}
%
%\end{document}
}
\subfigure[$\alpha/ \mu=1$ \label{fig:cook_alpha_1mu}]
{%\documentclass{article}
%
%\usepackage{graphicx}
%\usepackage{xcolor}
%\usepackage{colortbl}
%\usepackage{fp}
%\usepackage{tikz, pgfplots}
%\usetikzlibrary{calc}
%\usetikzlibrary{patterns}
%\usetikzlibrary{intersections}
%\usetikzlibrary{arrows}
%\tikzset{>=latex}
%
%% Color definition
%\definecolor{green2}{RGB}{154, 205, 50}
%\definecolor{green3}{RGB}{141, 182, 0}
%\definecolor{blue2}{RGB}{0, 102, 255}
%\definecolor{lightgreen}{RGB}{178,255,102}
%
%
%\begin{document}
%
%\begin{figure}[!h]
%\begin{center}
\begin{tikzpicture}
 %transpose legend
 \begin{axis}[width=0.45\textwidth, height=0.45\textwidth,
  legend style={anchor=south, at={(0.5, 1.07)}, draw=none, font=\scriptsize},
  legend columns=2,
  xlabel={Number of Elements per Side},
  ylabel={Vertical displacement of point A},
  xmin=0, xmax=32,
  ymin=0, ymax=8,
  ytick={0, 1, 2, 3, 4, 5, 6, 7, 8},
  yticklabels={$0$, $1$, $2$, $3$, $4$, $5$, $6$, $7$, $8$},
  xtick={0, 2, 4, 8, 16, 32},
  xticklabels={$0$, $2$, $4$, $8$, $16$, $32$},
 ]
 % Reference
 %\addplot[black, ultra thick] coordinates{
 %(50, 1.8248e-5)
 %(15000, 1.8248e-5)
 %}; 
 %\addlegendentry{Reference} 
 %
 \addplot[red, mark=+, thick, dashed, mark options={solid}]
 table [x index={0}, y index={1}]
 {quad_cook_membrane_graph/alpha_2mu/cook_1B_type_1.txt};
 \addlegendentry{1 Bubble type 1}
 %
 \addplot[red, mark=+, thick, mark options={solid}]
 table [x index={0}, y index={1}]
 {quad_cook_membrane_graph/alpha_2mu/cook_1B_type_2.txt};
 \addlegendentry{1 Bubble type 2}
 %
 \addplot[green3, mark=o, thick, mark options={solid}]
 table [x index={0}, y index={1}]
 {quad_cook_membrane_graph/alpha_2mu/cook_2B.txt};
 \addlegendentry{2 Bubble}
 %
 \addplot[blue, mark=x, thick, mark options={solid}]
 table [x index={0}, y index={1}]
 {quad_cook_membrane_graph/alpha_2mu/cook_2B_mixed.txt};
 \addlegendentry{2 Bubble Mixed}
 %
 %\addplot[blue, mark=x, very thick]
 %table [x index={0}, y index={1}]
 %{};
 %\addlegendentry{}
 %
 %\addplot[green3, mark=x, very thick]
 %table [x index={0}, y index={1}]
 %{};
 %\addlegendentry{}
 % 
 \end{axis}
\end{tikzpicture}
%\end{center}
%\caption{Vertical Displacement of point A vs. the number of element per side ($\alpha / \mu=2$)}
%\end{figure}
%
%\end{document}
}
\subfigure[$\alpha/ \mu=2$ \label{fig:cook_alpha_2mu}]
{%\documentclass{article}
%
%\usepackage{graphicx}
%\usepackage{xcolor}
%\usepackage{colortbl}
%\usepackage{fp}
%\usepackage{tikz, pgfplots}
%\usetikzlibrary{calc}
%\usetikzlibrary{patterns}
%\usetikzlibrary{intersections}
%\usetikzlibrary{arrows}
%\tikzset{>=latex}
%
%% Color definition
%\definecolor{green2}{RGB}{154, 205, 50}
%\definecolor{green3}{RGB}{141, 182, 0}
%\definecolor{blue2}{RGB}{0, 102, 255}
%\definecolor{lightgreen}{RGB}{178,255,102}
%
%
%\begin{document}
%
%\begin{figure}[!h]
%\begin{center}
\begin{tikzpicture}
 %transpose legend
 \begin{axis}[width=0.45\textwidth, height=0.45\textwidth,
  legend style={anchor=south, at={(0.5, 1.07)}, draw=none, font=\scriptsize},
  legend columns=2,
  xlabel={Number of Elements per Side},
  ylabel={Vertical displacement of point A},
  xmin=0, xmax=32,
  ymin=0, ymax=8,
  ytick={0, 1, 2, 3, 4, 5, 6, 7, 8},
  yticklabels={$0$, $1$, $2$, $3$, $4$, $5$, $6$, $7$, $8$},
  xtick={0, 2, 4, 8, 16, 32},
  xticklabels={$0$, $2$, $4$, $8$, $16$, $32$},
 ]
 % Reference
 %\addplot[black, ultra thick] coordinates{
 %(50, 1.8248e-5)
 %(15000, 1.8248e-5)
 %}; 
 %\addlegendentry{Reference} 
 %
 \addplot[red, mark=+, thick, dashed, mark options={solid}]
 table [x index={0}, y index={1}]
 {quad_cook_membrane_graph/alpha_2mu/cook_1B_type_1.txt};
 \addlegendentry{1 Bubble type 1}
 %
 \addplot[red, mark=+, thick, mark options={solid}]
 table [x index={0}, y index={1}]
 {quad_cook_membrane_graph/alpha_2mu/cook_1B_type_2.txt};
 \addlegendentry{1 Bubble type 2}
 %
 \addplot[green3, mark=o, thick, mark options={solid}]
 table [x index={0}, y index={1}]
 {quad_cook_membrane_graph/alpha_2mu/cook_2B.txt};
 \addlegendentry{2 Bubble}
 %
 \addplot[blue, mark=x, thick, mark options={solid}]
 table [x index={0}, y index={1}]
 {quad_cook_membrane_graph/alpha_2mu/cook_2B_mixed.txt};
 \addlegendentry{2 Bubble Mixed}
 %
 %\addplot[blue, mark=x, very thick]
 %table [x index={0}, y index={1}]
 %{};
 %\addlegendentry{}
 %
 %\addplot[green3, mark=x, very thick]
 %table [x index={0}, y index={1}]
 %{};
 %\addlegendentry{}
 % 
 \end{axis}
\end{tikzpicture}
%\end{center}
%\caption{Vertical Displacement of point A vs. the number of element per side ($\alpha / \mu=2$)}
%\end{figure}
%
%\end{document}
}
\subfigure[$\alpha/ \mu=3$ \label{fig:cook_alpha_3mu}]
{%\documentclass{article}
%
%\usepackage{graphicx}
%\usepackage{xcolor}
%\usepackage{colortbl}
%\usepackage{fp}
%\usepackage{tikz, pgfplots}
%\usetikzlibrary{calc}
%\usetikzlibrary{patterns}
%\usetikzlibrary{intersections}
%\usetikzlibrary{arrows}
%\tikzset{>=latex}
%
%% Color definition
%\definecolor{green2}{RGB}{154, 205, 50}
%\definecolor{green3}{RGB}{141, 182, 0}
%\definecolor{blue2}{RGB}{0, 102, 255}
%\definecolor{lightgreen}{RGB}{178,255,102}
%
%
%\begin{document}
%
%\begin{figure}[!h]
%\begin{center}
\begin{tikzpicture}
 %transpose legend
 \begin{axis}[width=0.45\textwidth, height=0.45\textwidth,
  legend style={anchor=south, at={(0.5, 1.07)}, draw=none, font=\scriptsize},
  legend columns=2,
  xlabel={Number of Elements per Side},
  ylabel={Vertical displacement of point A},
  xmin=0, xmax=32,
  ymin=0, ymax=8,
  ytick={0, 1, 2, 3, 4, 5, 6, 7, 8},
  yticklabels={$0$, $1$, $2$, $3$, $4$, $5$, $6$, $7$, $8$},
  xtick={0, 2, 4, 8, 16, 32},
  xticklabels={$0$, $2$, $4$, $8$, $16$, $32$},
 ]
 % Reference
 %\addplot[black, ultra thick] coordinates{
 %(50, 1.8248e-5)
 %(15000, 1.8248e-5)
 %}; 
 %\addlegendentry{Reference} 
 %
 \addplot[red, mark=+, thick, dashed, mark options={solid}]
 table [x index={0}, y index={1}]
 {quad_cook_membrane_graph/alpha_2mu/cook_1B_type_1.txt};
 \addlegendentry{1 Bubble type 1}
 %
 \addplot[red, mark=+, thick, mark options={solid}]
 table [x index={0}, y index={1}]
 {quad_cook_membrane_graph/alpha_2mu/cook_1B_type_2.txt};
 \addlegendentry{1 Bubble type 2}
 %
 \addplot[green3, mark=o, thick, mark options={solid}]
 table [x index={0}, y index={1}]
 {quad_cook_membrane_graph/alpha_2mu/cook_2B.txt};
 \addlegendentry{2 Bubble}
 %
 \addplot[blue, mark=x, thick, mark options={solid}]
 table [x index={0}, y index={1}]
 {quad_cook_membrane_graph/alpha_2mu/cook_2B_mixed.txt};
 \addlegendentry{2 Bubble Mixed}
 %
 %\addplot[blue, mark=x, very thick]
 %table [x index={0}, y index={1}]
 %{};
 %\addlegendentry{}
 %
 %\addplot[green3, mark=x, very thick]
 %table [x index={0}, y index={1}]
 %{};
 %\addlegendentry{}
 % 
 \end{axis}
\end{tikzpicture}
%\end{center}
%\caption{Vertical Displacement of point A vs. the number of element per side ($\alpha / \mu=2$)}
%\end{figure}
%
%\end{document}
}
\caption{Vertical Displacement of point A vs. 
the number of elements per side}
\end{center}
\end{figure}
All elements return different behaviour using different coefficients $\alpha$. In the case of $\alpha=1$, figure \ref{fig:cook_alpha_1}, the obtained results completely not converge to the reference solution.    

\end{document}